%----------------------------------------------------------------------------------------
%	CONFIGURATIONS
%----------------------------------------------------------------------------------------

\documentclass[12pt,a4paper,oneside]{article}

\usepackage[utf8]{inputenc}
\usepackage{graphicx}
\usepackage{epstopdf}
\usepackage{natbib}
\usepackage{amsmath}
\usepackage{lipsum}
\usepackage{caption}
\usepackage{subcaption}
\usepackage[a4paper,left=2cm,right=2cm,top=2.5cm,bottom=2.5cm]{geometry}

%----------------------------------------------------------------------------------------
%	INFORMATION
%----------------------------------------------------------------------------------------

\title{Programação Lógica para Problemas de Pesquisa e Análise Sintática da Gramática Portuguesa\\
  \vspace{0.1in}
  \large{Inteligência Artificial - Trabalho 3}
}

\author{João Ramos\footnote{João Ramos - 201204672} e Pedro Paredes\footnote{Pedro Paredes - 201205725}, DCC - FCUP}

\date{Maio 2015}

\renewcommand{\tablename}{Tabela}
\renewcommand{\figurename}{Figura}
\renewcommand{\refname}{Referências}

\begin{document}

\maketitle

%----------------------------------------------------------------------------------------
%	SECTION 1
%----------------------------------------------------------------------------------------

\section{Introdução}
\label{sec:intro}

O paradigma lógico tem um nível de abstração que permite abordagens e
representações diferentes para vários problemas. Atualmente, existe
uma vasta investigação na área em diversos temas, principalmente
usando a linguagem \texttt{Prolog}.

Neste trabalho iremos comparar a implementação de soluções em
\texttt{Prolog} com diferentes linguagens imperativas em diferentes
tipos de problemas. O primeiro tipo de problema será um problema de
pesquisa que envolve diferentes tipos de perguntas que requerem uma
aproximação semelhante à desenvolvida no trabalho 1. O segundo tipo de
problema é de interpretar uma gramática para reconhecimento de frases
na língua portuguesa.

O resto do relatório está organizado da seguinte forma. Na Secção
\ref{sec:pes} descrevem-se as implementações relativas ao problema de
pesquisa. Na Secção \ref{sec:sin} descrevem-se as implementações
relativas ao problema da gramática. Finalmente na Secção
\ref{sec:conc} fazem-se algumas notas finais.

%----------------------------------------------------------------------------------------
%	SECTION 1
%----------------------------------------------------------------------------------------

\section{Problemas de pesquisa}
\label{sec:pes}

O primeiro problema pede que se responda a três tipos de perguntas:

\begin{itemize}
\item Em que dias da semana há um voo direto entre duas localizações
  determinadas?
\item Que rotas existem que permitem chegar de uma determinada
  localização a outra determinada localização num dado dia,
  respeitando um tempo mínimo de 40 minutos entre cada voo?
\item Que rotas existem que permitam visitar uma dada lista de
  cidades, começando e terminando a viagem numa mesma dada cidade,
  iniciando a viagem num dado dia e fazendo um voo por dia?
\end{itemize}

\textbf{Nota:} A última questão foi alterada ligeiramente do enunciado
visto que a pergunta era um pouco ambígua e por isso levava a
diferentes interpretações (por exemplo, se a restrição é no máximo um
por dia, então isso seria equivalente a não ter restrição no sentido
do problema, pois poder-se-ia simplesmente voar no mesmo dia da semana
seguinte). Visto que a dificuldade da pergunta é a mesma, fixá-la
ajudou-nos a organizar as implementações.

\subsection{Implementação em \texttt{Prolog}}

Na implementação em \texttt{Prolog}

\lipsum[2]

\lipsum[3]

\subsection{Implementação em \textit{Ruby}}

\lipsum[1]

\lipsum[2]

\lipsum[3]

\subsection{Notas das Implementações}

\lipsum[1]

\lipsum[2]

%----------------------------------------------------------------------------------------
%	SECTION 2
%----------------------------------------------------------------------------------------

\section{Análise Sintática da Gramática Portuguesa}
\label{sec:sin}

\lipsum[1]

\subsection{Implementação em \texttt{Prolog}}

\lipsum[1]

\lipsum[2]

\lipsum[3]

\subsection{Implementação em \texttt{Python}}

\lipsum[1]

\lipsum[2]

\lipsum[3]

\subsection{Notas das Implementações}

\lipsum[1]

\lipsum[2]

%----------------------------------------------------------------------------------------
%	SECTION 3
%----------------------------------------------------------------------------------------

\section{Conclusão}
\label{sec:conc}

\lipsum[1]

\lipsum[2]

\bibliographystyle{plain}
\bibliography{lab_report_3}

\end{document}
