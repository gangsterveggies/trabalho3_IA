%----------------------------------------------------------------------------------------
%	CONFIGURATIONS
%----------------------------------------------------------------------------------------

\documentclass[12pt,a4paper,oneside]{article}

\usepackage[utf8]{inputenc}
\usepackage{graphicx}
\usepackage{epstopdf}
\usepackage{natbib}
\usepackage{amsmath}
\usepackage{lipsum}
\usepackage{caption}
\usepackage{subcaption}
\usepackage[a4paper,left=2cm,right=2cm,top=2.5cm,bottom=2.5cm]{geometry}

%----------------------------------------------------------------------------------------
%	INFORMATION
%----------------------------------------------------------------------------------------

\title{Programação Lógica para Problemas de Pesquisa e Análise Sintática da Gramática Portuguesa\\
  \vspace{0.1in}
  \large{Inteligência Artificial - Trabalho 3}
}

\author{João Ramos\footnote{João Ramos - 201204672} e Pedro Paredes\footnote{Pedro Paredes - 201205725}, DCC - FCUP}

\date{Maio 2015}

\renewcommand{\tablename}{Tabela}
\renewcommand{\figurename}{Figura}
\renewcommand{\refname}{Referências}

\begin{document}

\maketitle

%----------------------------------------------------------------------------------------
%	SECTION 1
%----------------------------------------------------------------------------------------

\section{Introdução}
\label{sec:intro}

O paradigma lógico tem um nível de abstração que permite abordagens e
representações diferentes para vários problemas. Atualmente, existe
uma vasta investigação na área em diversos temas, principalmente
usando a linguagem \texttt{Prolog}.

Neste trabalho iremos comparar a implementação de soluções em
\texttt{Prolog} com diferentes linguagens imperativas em diferentes
tipos de problemas. O primeiro tipo de problema será um problema de
pesquisa que envolve diferentes tipos de perguntas que requerem uma
aproximação semelhante à desenvolvida no trabalho 1. O segundo tipo de
problema é de interpretar uma gramática para reconhecimento de frases
na língua portuguesa.

O resto do relatório está organizado da seguinte forma. Na Secção
\ref{sec:pes} descrevem-se as implementações relativas ao problema de
pesquisa. Na Secção \ref{sec:sin} descrevem-se as implementações
relativas ao problema da gramática. Finalmente na Secção
\ref{sec:conc} fazem-se algumas notas finais.

%----------------------------------------------------------------------------------------
%	SECTION 1
%----------------------------------------------------------------------------------------

\section{Problemas de pesquisa}
\label{sec:pes}

O primeiro problema pede que se responda a três tipos de perguntas:

\begin{itemize}
\item Em que dias da semana há um voo direto entre duas localizações
  determinadas?
\item Que rotas existem que permitem chegar de uma determinada
  localização a outra determinada localização num dado dia,
  respeitando um tempo mínimo de 40 minutos entre cada voo?
\item Que rotas existem que permitam visitar uma dada lista de
  cidades, começando e terminando a viagem numa mesma dada cidade,
  iniciando a viagem num dado dia e fazendo um voo por dia?
\end{itemize}

\textbf{Nota:} A última questão foi alterada ligeiramente do enunciado
visto que a pergunta era um pouco ambígua e por isso levava a
diferentes interpretações (por exemplo, se a restrição é no máximo um
por dia, então isso seria equivalente a não ter restrição no sentido
do problema, pois poder-se-ia simplesmente voar no mesmo dia da semana
seguinte). Visto que a dificuldade da pergunta é a mesma, fixá-la
ajudou-nos a organizar as implementações.

\subsection{Implementação em \texttt{Prolog}}

Na implementação em \texttt{Prolog} usámos dois ficheiros com
predicados. O primeiro, \texttt{main.pl} contém os predicados os
predicados que respondem às questões anteriores e vários predicados
auxiliares. Adicionalmente, temos um ficheiro \texttt{bd.pl} que
contém as entradas na base de dados de voos (na forma de predicados
\texttt{timetable/3}. Os predicados que respondem às questões
anteriores são os seguintes.

Para a primeira questão, temos o predicado \texttt{directflight/3}, que
aceita o cabeçalho \texttt{directflight( +Place1, +Place2, -Day )}, onde
\texttt{Place1} representa a localização inicial, \texttt{Place2}
representa a localização final e a resposta que o predicado dá é o
\texttt{Day}, que representa cada dia que satisfaz as condições da
pergunta.

Para a segunda questão, temos o predicado \texttt{route/4}, que aceita
o cabeçalho \texttt{route( +Place1, +Place2, ?Day, -Route )}, onde
\texttt{Place1} representa a localização inicial, \texttt{Place2}
representa a localização final, \texttt{Day} é um argumento opcional
que representa o dia da viagem e finalmente a resposta estará em
\texttt{Route}, que representa a rota da viagem no formato dado no
enunciado do trabalho.

Finalmente, a terceira questão, temos o predicado \texttt{visit/4},
que aceita o cabeçalho \texttt{visit( +Placei, +Day, +Destlist,
  -Route )}, onde \texttt{Placei} representa a localização inicial e
final, \texttt{Day} é representa o dia em que começa a viagem,
\texttt{Destlist} é uma lista de destinos onde se pretende passar e
finalmente a resposta estará em \texttt{Route}, que representa a rota
da viagem no formato dado no enunciado do trabalho.

Além disto, foi implementado um programa em \texttt{C} que utiliza o
\textit{foreign interface} do \texttt{Prolog} para comunicar, no
ficheiro \texttt{main.c}. Para compilar este código deve-se usar o
comando \texttt{swipl-ld -o main main.c main.pl bd.pl}. Este pequeno
programa apenas introduz um menu que permite chamar os vários
predicados para responder às perguntas interativamente.

\subsection{Implementação em \textit{Ruby}}

Na implementação em \texttt{Ruby} usámos apenas um ficheiro que
responde às três perguntas, juntamente com um menu semelhante ao usado
no programa em \texttt{C} que utiliza o \texttt{Prolog} para responder
às perguntas. Os vôos são armazenados numa \textit{hash table} para
poderem ser obtidos de forma eficiente quando necessários.

Para a primeira questão, a função responsável por responder à mesma
apenas percorre a lista de vôos entre as cidades dadas, imprimindo
informações dos mesmos à medida que é feita a união dos conjuntos de
dias em que cada vôo está disponível.

Para a segunda questão, é feita uma pesquisa em profundidade mantendo
no estado as cidades já visitadas para não voltar às mesmas, e imprimir
informação de cada rota no final. Nota para que apenas são pesquisados
espaços de procura válidos. Caso apenas estivéssemos interessados em
obter uma das rotas (a mais rápida por exemplo) podiamos ter usado um
algoritmo diferente (algoritmo de \textit{Dijkstra} por exemplo).

Finalmente, a terceira questão, são geradas todas as possíveis ordem de
visita das cidades e, para cada uma, é verificado se a mesma é possível
utilizando a função responsável por responder à primeira pergunta.

\subsection{Notas das Implementações}

Neste problema, a representação em \texttt{Prolog} fica bastante
natural, pois visto que o funcionamento do \texttt{Prolog} é baseado
em \textit{backtracking} (além de outros conceitos, como unificação,
claro). Porém, a eficiência da segunda pergunta podia ser bastante
otimizada com algoritmos mais específicos para grafos, como referido
na subsecção anterior, cuja representação em \texttt{Prolog} não é tão
natural (apesar de ser possível).

Dito isto, os operadores de \texttt{Prolog} em conjunto com as
definições naturais dos predicados ajudam a que a implementação em
\texttt{Prolog} fique mais simplificada, facilitando a sua
implementação.

%----------------------------------------------------------------------------------------
%	SECTION 2
%----------------------------------------------------------------------------------------

\section{Análise Sintática da Gramática Portuguesa}
\label{sec:sin}

Para o segundo problema, é pedido que se construa um programa que
analise frases em português, indicando se são válidas ou não e
devolvendo a \textit{parse tree} da frase segundo estruturas
sintáticas. Apenas foi considerada a sintaxe das frase, tendo sido
ignorada a semântica, como indicado no enunciado.
 
\subsection{Gramática utilizada}

A gramática que foi implementada é baseada num livro de gramática
portuguesa \cite{arezedo:2013}. Existe alguma margem para interpretação
de como deve ser estruturada a gramática portuguesa, por isso seguimos
o mais possível o livro indicado.

Apesar de apenas ser necessário considerar a sintaxe da frase, a
introdução do género e número acaba por pertencer ao domínio da
semântica. Assim, para manter a implementação o mais limpa possível,
estes foram passados como argumentos em ambas as implementações e a
sua concordância foi garantida a nível da gramática. Uma implementação
mais eficiente ignoraria estes fatores e posteriormente faria uma
análise semântica ao percorrer a \textit{parse tree} e garantido a
concordância de todos os atributos (o que poderia incluir outros
atributos, como a pessoa, tempo verbal, \ldots). Além disto,
introduzimos a noção de verbo intransitivo e verbo transitivo (que
pode ser transitivo direto ou transitivo indireto) na gramática e
consideramos estes fatores na \textit{parse tree} (apesar de ser algo
que também pertence à semântica).

Consideramos também frases simples e complexas, usando conjunções
coordenadas (a subordinação implica a mudança da estrutura da frase e
é um exemplo onde a semântica altera a sintaxe e que complicaria ainda
mais esta gramática). Isto não se aplica ao nível de coordenação de
grupos nominais (por exemplo, ``Eu e o João corremos''), pois alterava
a concordância de número para plural e complicava assim a
implementação (o grupo nominal e o grupo verbal deixariam de ser
independentes em termos de conteúdo).

Uma outra nota importante é que para considerarmos contrações de
palavras (por exemplo, ``na'' = ``em'' + ``a''), usamos um conceito
fictício de preposições pronomiais e preposições determinantes,
dependendo se a contração envolve uma preposição e um pronome ou uma
preposição e um determinante. Pelo que sabemos este conceito não
existe na literatura linguística (apesar de existir na computacional),
mas simplifica-nos a representação.

Isto tudo resulta numa gramática que permite reconhecer as frases de
exemplo do enunciado, mas também várias outras formas
diferentes. Claro que ignoramos muitos casos importantes da gramática
portuguesa, sendo os mais importantes a forma passiva e os
predicativos de sujeito (que são usados com verbos específicos como
``ser'').

As bases de dados que colocámos não são muito extensas, apenas incluem
as palavras necessárias para reconhecer todas as frases de exemplo do
enunciado. Tínhamos uma base de dados maior de palavras que estávamos
a considerar em incluir, infelizmente era necessário fazer algumas
modificações ao seu conteúdo (por exemplo, os verbos não diziam se
eram transitivos ou intransitivos) \cite{label:2003} e por falta de
tempo não concluímos a sua inserção.

\subsection{Implementação em \texttt{Prolog}}

A implementação em \texttt{Prolog} usa o conceito de \textit{Definite
  Clause Grammars} que por sua vez é implementado em \texttt{Prolog}
usando diferença de listas. Esta é uma técnica elegante para
representar uma gramática que é muito fácil de introduzir em
\texttt{Prolog}.

Na implementação temos a considerar 4 ficheiros diferentes. O
\texttt{bd.pl} contém o dicionário de palavras a serem utilizadas
pelos predicados da gramática. O seu formato foi feito para
simplificar a gramática e fará mais sentido após ver o resto da
implementação. O ficheiro \texttt{synt\_gen.pl} contém a gramática
inicial que apenas testa se uma frase está correta ou não (não gera a
\textit{parse tree}. O ficheiro \texttt{synt\_gen\_parse.pl} contém
uma cópia do ficheiro anterior, mas introduz argumentos extra nos
predicados de forma a gerar a \textit{parse tree}. Finalmente o
ficheiro \texttt{test\_bed.pl} contém apenas as frases de exemplo do
enunciado para propósitos de teste.

Para diminuir a complexidade da gramática, usam-se argumentos extra em
cada tipo sintático representando os vários atributos, como referido
na subsecção anterior. A concordância destes atributos é testada nos
predicados ``nucleares'' da gramática (que representam elementos
sintáticos simples, como o predicado \texttt{verbo/4}). Assim, o
dicionário de palavras guarda, além das próprias palavras, os tipos
que representam assim como outros atributos (como o número de
género). O caso mais extremo é o caso dos verbos, que para evitar
repetição e entradas no dicionário, cada verbo conjugado inclui um
atributo com a sua forma verbal infinitiva. Para garantir a
concordância do tipo de verbo (transitivo, intransitivo), utiliza-se a
forma verbal infinitiva que contém uma entrada extra no dicionário com
esta informação.

Além disto, foi implementado um programa em \texttt{C} que utiliza o
\textit{foreign interface} do \texttt{Prolog} para comunicar, no
ficheiro \texttt{main.c}. Este implementa um menu simples que permite
verificar a correção de frases assim como gerar as devidas
\textit{parse trees} interativamente.

\subsection{Implementação em \texttt{Python}}

Na implementação em \texttt{Python} utilizámos o módulo \texttt{nltk} e
usámos dois ficheiros. O primeiro, \texttt{pt\_grammar.fcfg}, contém a nossa
gramática da língua portuguesa no formato aceite pelo \texttt{nltk}, assim como
a base de dados de palavras com a sua respectiva categorização lexicográfica.
E temos o ficheiro \texttt{dcg.py} que implementa a função \texttt{sentence\_analysis}
que recebe a frase a analisar como argumento e, caso a mesma esteja correcta, mostra
a análise morfológica da mesma.

A gramática utilizada é a mesma introduzida no início desta secção. Assim com a base
de dados utilizada.

Para a implementação da função \texttt{sentence\_analysis} começamos por carregar a
nossa gramática para o parser e fazer um pequeno tratamento da frase recebida (ignorar
sinais de pontuação, converter todos os caracteres para minúsculas e dividir a frase
na lista das palavras que a compõe). Após isto, o \texttt{nltk} usa a nossa gramática
para gerar as árvores da análise sintática da frase recebida.

Implementámos ainda a função \texttt{assert\_sentence\_analysis} que tem como objectivo
testar a nossa função de análise sintática com os exemplos fornecidos no enunciado do
trabalho.

\subsection{Notas das Implementações}

Neste problema, as representações em \texttt{Prolog} e \texttt{Python} ficam ambas bastante
naturais e parecidas graças ao módulo \texttt{nltk} utilizado no python que permite uma
implementação simples mas bastante poderosa da gramática.

Ambas as implementações, particularmente a implementação em \texttt{Python} graças às
potencialidades oferecidas pelo \texttt{nltk}, permitem melhorias significativas não só a
nível de eficiência como da qualidade da análise sintática mantendo as implementações com
baixo nível de complexidade.


%----------------------------------------------------------------------------------------
%	SECTION 3
%----------------------------------------------------------------------------------------

\section{Conclusão}
\label{sec:conc}

Este trabalho permitiu explorar e ver em prática a utilização de uma linguagem com um
diferente paradigma para a resolução de problemas com características diferentes.

O paradigma lógico permite, nos problemas analisados, implementações mais concisas e
simples, mesmo sem ser necessário recorrer a módulos extra, como foi o caso na implementação
do segundo problema numa linguagem imperativa em que, para ter uma implementação simples
mas poderosa foi necessário utilizar um módulo extra.

Apesar disto, há situações e problemas em que a situação se inverte, ou seja, em que é mais
simples a implementação numa linguagem imperativa em relação a uma linguagem lógica. Contudo,
é sobre bom ter mais ``armas'' à disposição para atacar cada problema da melhor maneira.


\bibliographystyle{plain}
\bibliography{lab_report_3}

\end{document}
